\documentclass[a4paper]{article}
\usepackage[utf8]{inputenc}
\usepackage[margin=1.5cm]{geometry}
\usepackage{booktabs}
\usepackage{array}
\usepackage{graphicx}
\usepackage{caption}
\usepackage{adjustbox}
\usepackage{caption}
\usepackage{subcaption}
\usepackage{siunitx}
\usepackage{longtable}
%\usepackage[table]{xcolor}

\captionsetup[table]{skip=5pt, font=footnotesize, justification=raggedright, singlelinecheck=false}
\renewcommand{\arraystretch}{1.2} % Slightly more vertical space

\begin{document}
	
	% two models of somatic mosaicism
	
	\begin{figure}[t]
		\centering
		\includegraphics[width=0.6\linewidth]{../../legacy/figures/mosaicism_models/brain-damage-combined.pdf}
		\caption{\textbf{Two models for somatic mosaicism in sporadic prion disease}. \textbf{Top left}~|~In the presymptomatic brain, \textit{PRNP} is mutated in an anatomically circumscribed island (purple region) derived from a clonal population that arose in a late stage of brain development. \textbf{Top right}~|~An initial prion seed develops within the island and propagates to non-mutated regions, producing progressive pathological spread (red gradient). \textbf{Bottom left}~|~Alternatively, \textit{PRNP}-mutant cells (purple dots) may be distributed throughout the brain, reflecting an early progenitor mutation or multiple independent, age-related events. \textbf{Bottom right}~|~Prion seeds emerge in one or multiple mutational foci and disseminate pathology to other brain areas.}
		\label{fig:prnp-mosaic}
	\end{figure}
	
	\clearpage
	
	
	% DNA purification and decontamination
	\begin{figure}[t]
		\centering
		\includegraphics[width=0.9\linewidth]{../../legacy/figures/purification_protocol/purification-overview-paper.pdf}		
		\caption{
			\textbf{DNA purification and decontamination.}
			Protease digestion: bulk brain tissue was lysed and digested overnight with proteinase K (PK). Prion disaggregation: a buffered guanidine–SCN solution (4 mol/l) was added to disaggregate residual PK-resistant prions. Phase separation: three sequential phenol–chloroform extractions separated DNA from RNA and proteins. DNA from the aqueous phase was precipitated with ethanol, washed in 70 \% ethanol, and purified using SPRI-bead clean-up to remove residual salts.
		}
		\medskip
		
		\clearpage
	\end{figure}
	
		% SNV analysis pipeline
	\begin{figure}[t]
		\centering
		\includegraphics[width=0.9\linewidth]{../snv_flowchart/snv_flowchart_tikz.pdf}
		\caption{\textbf{Analytical workflow for single-nucleotide variant (SNV) detection.}. Paired-end sequencing reads underwent quality trimming, alignment with BWA-MEM, sorting and indexing using SAMtools, duplicate marking with Picard, and base-quality recalibration via GATK BaseRecalibrator. Somatic variants were identified with Mutect2, followed by normalisation, functional annotation, and integration of population allele-frequency data. Read- and base-level metrics were incorporated into a custom R workflow applying established quality filters for read depth, strand representation, base and mapping quality, population allele frequency, and minimum variant-allele fraction. Blue boxes: processes; green boxes: data outputs; orange boxes: bioinformatics tools; red box: final dataset.}
		\medskip
	\end{figure}
	
	\clearpage
	
	
	
	
	% RT-QuIC of decontaminated DNA
	
	\begin{figure}[t]
		\centering
		
		% --- Row 1 ---
		\begin{subfigure}[t]{0.3\textwidth}
			\centering
			\includegraphics[width=\linewidth]{../../legacy/figures/rt-quic/unseeded.pdf}
			\caption{Unseeded}
			\label{subfig:RT-QuIC-a}
		\end{subfigure}\hfill
		\begin{subfigure}[t]{0.3\textwidth}
			\centering
			\includegraphics[width=\linewidth]{../../legacy/figures/rt-quic/CJD_CB.pdf}
			\caption{CJD cerebellum}
			\label{subfig:RT-QuIC-b}
		\end{subfigure}\hfill
		\begin{subfigure}[t]{0.3\textwidth}
			\centering
			\includegraphics[width=\linewidth]{../../legacy/figures/rt-quic/noninfect_CB.pdf}
			\caption{Non-infectious cerebellum}
			\label{subfig:RT-QuIC-c}
		\end{subfigure}
		
		\vspace{1.2ex}
		
		% --- Row 2 ---
		\begin{subfigure}[t]{0.3\textwidth}
			\centering
			\includegraphics[width=\linewidth]{../../legacy/figures/rt-quic/CJD_Trizol_RNA.pdf}
			\caption{Purified CJD RNA}
			\label{subfig:RT-QuIC-d}
		\end{subfigure}\hfill
		\begin{subfigure}[t]{0.3\textwidth}
			\centering
			\includegraphics[width=\linewidth]{../../legacy/figures/rt-quic/CJD_deko_DNA.pdf}
			\caption{Purified CJD gDNA}
			\label{subfig:RT-QuIC-e}
		\end{subfigure}\hfill
		\begin{subfigure}[t]{0.3\textwidth}
			\centering
			\includegraphics[width=\linewidth]{../../legacy/figures/rt-quic/noninfect_DNA.pdf}
			\caption{Purified non-infectious gDNA}
			\label{subfig:RT-QuIC-f}
		\end{subfigure}
		
		\caption{\textbf{Real-time quaking-induced conversion (RT-QuIC) of purified CJD brain nucleic acids indicates successful decontamination.}
			(a) Unseeded control; (b) unpurified CJD cerebellum shows prion seeding; (c) non-infectious cerebellum shows no prion seeding; (d–e) RNA (d) and gDNA (e) purified from CJD brain show no prion seeding; (f) gDNA purified from non-infectious brain shows no RT-QuIC signal. RFU: relative fluorescence units.}
		\label{fig:RT-QuIC}
	\end{figure}
	
	\medskip
	\clearpage
	
	
	
	
% ddPCR LoD summary — three mutations × two metrics (3 rows × 2 columns)
\begin{figure}[t]
	\centering
	
	% ----- Row 1: D178N -----
	\begin{subfigure}[t]{0.47\textwidth}
		\centering
		\includegraphics[width=\linewidth]{../ddpcr_lod/D178N_LoD_concentration.pdf}
		\caption{D178N concentration}
		\label{subfig:LoD-D178N-conc}
	\end{subfigure}\hfill
	\begin{subfigure}[t]{0.47\textwidth}
		\centering
		\includegraphics[width=\linewidth]{../ddpcr_lod/D178N_fraction.pdf}
		\caption{D178N fractional abundance}
		\label{subfig:LoD-D178N-frac}
	\end{subfigure}
	
	\vspace{1.5ex}
	
	% ----- Row 2: E200K -----
	\begin{subfigure}[t]{0.47\textwidth}
		\centering
		\includegraphics[width=\linewidth]{../ddpcr_lod/E200K_LoD_concentration.pdf}
		\caption{E200K concentration}
		\label{subfig:LoD-E200K-conc}
	\end{subfigure}\hfill
	\begin{subfigure}[t]{0.47\textwidth}
		\centering
		\includegraphics[width=\linewidth]{../ddpcr_lod/E200K_fraction.pdf}
		\caption{E200K fractional abundance}
		\label{subfig:LoD-E200K-frac}
	\end{subfigure}
	
	\vspace{1.5ex}
	
	% ----- Row 3: P102L -----
	\begin{subfigure}[t]{0.47\textwidth}
		\centering
		\includegraphics[width=\linewidth]{../ddpcr_lod/P102L_LoD_concentration.pdf}
		\caption{P102L concentration}
		\label{subfig:LoD-P102L-conc}
	\end{subfigure}\hfill
	\begin{subfigure}[t]{0.47\textwidth}
		\centering
		\includegraphics[width=\linewidth]{../ddpcr_lod/P102L_fraction.pdf}
		\caption{P102L fractional abundance}
		\label{subfig:LoD-P102L-frac}
	\end{subfigure}
	
	\caption{\textbf{Limit-of-detection (LoD) testing for three PRNP mutations (D178N, E200K, P102L).}
		(a–b) D178N, (c–d) E200K, (e–f) P102L. Left panels show allele concentrations for mutant and wild-type (WT) alleles across NTC, WT DNA, and artificial mosaics; right panels show the corresponding fractional abundances. Red points denote mutant copy number; turquoise points denote WT allele copy number. The calculated values matched expected concentrations, with 0.05 \% mosaics clearly distinguishable from NTC and WT controls. Error bars: 95 \% Poisson confidence intervals; cpm: copies per µl; dashed blue line: limit of blank (LoB).}
	\label{fig:LoD-summary}
\end{figure}

	\medskip
	\clearpage

% ddPCR results

\begin{figure}[t]
	\centering
	\includegraphics[width=0.9\linewidth]{../ddpcr_fractional_abundance/SNV_all_mutations_legend_bottom_final.pdf}
	\caption{
		\textbf{Mutant allele frequencies in CJD and control brain samples.}
		Bulk DNA was extracted from post-mortem brain tissue of patients with sporadic CJD and non-prion disease controls.
		The DNA was analysed by droplet digital PCR (ddPCR) to assess the presence of the pathogenic \textit{PRNP} mutations D178N (top), E200K (middle) and P102L (bottom).
		Each point represents the estimated mutant allele frequency (MAF) for an individual brain sample, calculated as the percentage of total DNA.
		Vertical bars denote 95\% confidence intervals derived from Poisson-based concentration estimates.
		The horizontal dashed line marks the Limit of detection (LoD) threshold for every sample.
		Most samples exhibited MAF values below this threshold.
		A few CJD samples showed apparent low-level signals (0.05--0.1\%) but these did not exceed the LoD or limit of blank (LoB) and therefore could not be distinguished from assay background.
		Colours indicate the anatomical brain region of origin. Filled circles: samples above LoB; open circles: samples below LoB.
	}
\end{figure}


	\medskip
	\clearpage
	
	
	% SNV lollipop graph
	\begin{figure}[t]
		\centering
		\includegraphics[width=0.9\linewidth]{../snv_lollipop/SNV_lollipop_final.pdf}
		\caption{\textbf{Detection of somatic intronic variants in prion disease brain tissue.}. A detection threshold of 0.8 \% variant allele frequency (VAF) was established through limit-of-detection experiments. In sample CJD23, an intronic variant (chr20:4694249 T{\textgreater}C) exceeded this threshold. Another variant (chr20:4691920 G{\textgreater}A) was detected in three cases (CJD2, CJD6, and CJD23) but remained below the detection limit despite fulfilling all other quality-control criteria. Node colour denotes sample identity. Variants detected below the limit of detection are shown with reduced colour intensity to indicate lower confidence. Horizontal axis: genomic coordinates within \textit{PRNP}. Vertical axis: VAF.}
		\label{fig:snv-lollipop}
		\medskip
	\end{figure}	
	
	\clearpage
	
	
	
	
	
\begin{figure}[t]
	\centering
	\includegraphics[width=0.9\linewidth]{../ddpcr_fractional_abundance_pooled/SNV_pooled_all_mutations_one_legend.pdf}
	\caption{
		\textbf{Mutant allele frequencies in pooled brain DNA from individual participants.}
		This post hoc subgroup analysis shows the variant allele frequencies (VAFs) of the \textit{PRNP} mutations D178N (A), E200K (B) and P102L (C) obtained after in silico pooling of droplet counts from multiple brain regions per participant. 
		Each point represents the estimated VAF per participant, with 95\% confidence intervals derived from Poisson-based concentration estimates. 
		The horizontal dashed line marks the limit of detection (LoD) for the respective assay.
		All pooled samples exhibited VAFs below this threshold; filled circles denote samples exceeding the limit of blank (LoB), open circles those not exceeding it.
	}
	\medskip
	\clearpage
\end{figure}


\end{document}
