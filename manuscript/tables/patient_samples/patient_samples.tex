\documentclass[a4paper,11pt]{article}

% Page/layout
\usepackage[margin=1.5cm]{geometry}

% Table + formatting
\usepackage{booktabs}   % \toprule, \bottomrule
\usepackage{array}      % \newcolumntype
\usepackage{graphicx}   % \resizebox
\usepackage{caption}    % \captionof
\usepackage{amsmath}    % \text{} in math mode
\usepackage{afterpage}  % \afterpage
\usepackage{pdflscape}  % landscape pages that rotate correctly in PDF viewers
\usepackage{xspace}     % insert space after macros

% Column type used in your tabular:
% 'L' = left-aligned math mode column
% 'C' = centered matho mode
\newcolumntype{L}{>{$}l<{$}}
\newcolumntype{C}{>{$}c<{$}}

% Minimal macro definitions used in your table/caption.
% If your manuscript already defines these differently, adjust here to match.
\newcommand{\prnp}{\textit{PRNP}\xspace}
\newcommand{\prpsc}{PrP\textsuperscript{Sc}\xspace}

% Struts used in your table for vertical spacing
\newcommand{\Tstrut}{\rule{0pt}{2.6ex}}
\newcommand{\Bstrut}{\rule[-1.2ex]{0pt}{4.0ex}}

\begin{document}
	
	%patient list
	\afterpage{%
		\clearpage% Flush earlier floats (otherwise order might not be correct)
		\thispagestyle{empty}% empty page style
		\begin{landscape}% Landscape page
			\centering % Center table
			\resizebox{1\linewidth}{!}{%
				\begin{tabular}{lCcClllcll} %age and duration in math mode. Put NA in \text{}.
					\toprule[2pt]\Bstrut
					name & \text{age} & sex & \text{time to death (d)} & symptoms & main diagnosis & other diagnoses & codon $129$ & histotype & \prpsc type \\
					\hline \Tstrut
					CJD1 & 66 & F & 95 & visual disturbances, ataxia, dystonia, EEG compatible with sCJD & sCJD &  & MM & MM1+2C & type 1 (2) \\
					CJD2 & 74 & M & 74 & mild dementia, myoclonus, nystagmus, ataxia, palsy of right arm & sCJD &  & MM & MM1 & type 1 \\
					CJD3 & 69 & M & 368 & dementia (cortical and subcortical) & sCJD &  & MM & atypical & type 2 \\
					CJD4 & 66 & F &  &  & sCJD &  & VV & VV2 & type 2 \\
					CJD5 & 71 & M & 543 & progressive fronto-subcortical dysfunction, visual disturbances, EEG compatible with sCJD & sCJD &  & MV & MV2K+C & type 2 (1) \\
					CJD6 & 62 & F &  &  & sCJD &  & MV & MV2C & type 2 \\
					CJD7 & 78 & F & 324 & moderate cognitive deficits, postural instability, sensory axonal polyneuropathy & sCJD &  & MV & atypical & type 1 \\
					CJD8 & 72 & M & 169 & severe cognitive deficits, memory loss, visual disturbances, myoclonus, ataxia, gait disorder & sCJD &  & MM & MM1 & type 1 \\
					CJD9 & 56 & M & 86 & cognitive deficits, memory loss, amnestic aphasia, non-convulsive status epilepticus & sCJD &  & MM & atypical & type 1 \\
					CJD10 & 73 & F & 121 & Broca's aphasia, mild cerebellar dysfunction & sCJD & intermed. AD, mild CAA & MM & MM1+2C & type 1 (2) \\
					CJD11 & 48 & F & 64 & memory loss, affective disorder, nystagmus, extrapyramidal symptoms, myoclonus & sCJD &  & MM & MM1 & type 1 \\
					CJD12 & 76 & M & 213 & rapidly progressive dementia, visual disturbances, gait disorder & sCJD &  & MM & MM1 & type 1 \\
					CJD13 & 57 & F & 100 & cognitive deficits, memory loss, amnestic aphasia, gait disorder & sCJD &  & MM & MM1 & type 1 \\
					CJD14 & 83 & M & 152 & dementia, myoclonus, visual disturbances, positive psychiatric symptoms & sCJD &  & MV & atypical & type 1 \\
					CJD15 & 63 & M & 95 & dementia, visual disturbances, cerebellar dysfunction, pyramidal signs, extrapyramidal symptoms & sCJD & low AD, moderate CAA & MM & MM+2C & type 1 \\
					CJD16 & 71 & F & 55 & mild cognitive deficits (loss of memory, concentration, impulse control), myoclonus, ataxia & sCJD & low AD & MM & MM1 & type 1 \\
					CJD17 & 67 & M &  &  & sCJD & low AD & MM & MM1+2C & type 1 \\
					CJD18 & 69 & F & 180 & myoclonus, pyramidal signs & sCJD & low AD & MV & MV2K+C & type 2 \\
					CJD19 & 59 & M & 363 & dementia, progressive dysarthria & sCJD & stroke & MV & atypical & types 1 \& 2 \\
					CJD20 & 57 & M & 156 & dementia, myoclonus, cerebellar dysfunction & sCJD &  & MM & MM1 & type 1 \\
					CJD21 & 69 & M & 131 & rapidly progressive dementia, ataxia, myoclonus, EEG compatible with sCJD & sCJD & low AD, moderate CAA & MM & MM1 & type 1 \\
					CJD22 & 72 & M & 55 & dysarthria, dysphagia, gait disorder & sCJD & PART, schizophrenia & MM & MM1 & type 1 \\
					CJD23 &  &  &  &  & sCJD &  & VV & VV1 & \\
					CJD24 &  &  &  &  & sCJD &  & MM & MM2T & \\
					CJD25 &  &  &  &  & sCJD &  & MM & MM2T & \\
					CJD26 &  &  &  &  & sCJD &  & MM & MM2T & \\
					CJD27 &  &  &  &  & sCJD &  & MV & MV2K & \\
					CJD28 &  &  &  &  & sCJD &  & MV & MV2K & \\
					CJD29 &  &  &  &  & sCJD &  & MV & MV2K & \\
					CJD30 & 67 &  &  &  & gCJD (E200K) &  & VV & VV2 & \\
					CJD31 & 74 &  &  &  & sCJD &  & VV & VV2 & \\
					\hline \Tstrut
					Control1 & 78 & F &  & dementia, Parkinsonism, multifocal epilepsy, EEG compatible with sCJD & high AD & CAA & MV &  &  \\
					Control2 & 72 & M &  & rapidly progressive neurological disorder, rigidity & multiple ischaemic strokes & atherosclerosis, PART & MV &  &  \\
					Control3 & 76 & M &  & rapidly progressive dementia, postural instability, cachexia & CMV encephalitis &  & MM &  &  \\
					Control4 & 78 & M &  & visual disturbances, dizziness, paralysis of eye muscles & severe alpha-synucleinopathy & PART & MM &  &  \\
					Control5 & 70 & F &  & dementia, Parkinsonism, temporal lobe atrophy & Lewy body disease & PART, TDP-43 & MV &  &  \\
					Control6 & 84 & F &  & ataxia, reduced vigilance, epilepsy, EEG compatible with sCJD & paraneoplastic encephalomyelitis & low AD & MM &  &  \\
					Control7 & 64 & M &  & dementia & FTLD with TDP-43 pathology & intermed. AD & MV &  &  \\
					Control8 & 78 & F &  & rapidly progressive dementia, cerebellar dysfunction, pyramidal signs, epilepsy & CAA & inflammatory encephalopathy & MV &  &  \\
					\bottomrule[2pt]
				\end{tabular}
			}
			\captionof{table}{\textbf{Overview of patients included in the study}. Age at death, sex, time from onset of symptoms to death (illness duration, given in days), main clinical symptoms, main neuropathological diagnosis, other neuropathological diagnoses, \prnp codon $129$ genotype, prion disease histotype and predominant \prpsc subtype (minor \prpsc subtype given in brackets, if present) are given, where available. In the control samples, other brain pathologies were identified. Post-mortem tissue was collected from every patient and biobanked. DNA was also purified from sorted frontal cortex nuclei of selected patients and underwent targeted \prnp sequencing. Low, intermediate, high AD: degree of AD-related neuropathological change. AD: Alzheimer's disease. CAA: cerebral amyloid angiopathy. CMV: cytomegalovirus. FTLD: frontotemporal lobar degeneration. PART: primary age-related tauopathy.}
			\label{tab:patients}
		\end{landscape}
		\clearpage% Flush page
	}
	
\end{document}
